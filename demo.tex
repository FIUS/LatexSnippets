\documentclass[12pt]{scrartcl}
\usepackage[utf8]{inputenc}
\usepackage[T1]{fontenc}
\usepackage[ngerman]{babel}
\usepackage{listings}
\usepackage{tikz}

\usetikzlibrary{shapes}
\usetikzlibrary{chains}
\usetikzlibrary{arrows}
 
\definecolor{lightYellow}{RGB}{255,255,204}
\definecolor{grayYellow}{RGB}{153,153,122}
\lstset{framexleftmargin=5mm, frame=shadowbox, rulesepcolor=\color{grayYellow}, rulecolor=\color{lightYellow}}
\lstset{backgroundcolor=\color{lightYellow}}


\begin{document}
  \section{Linked List}
  \begin{figure}[!h]
  \centering
  \begin{tikzpicture}[start chain, node distance=10mm]
    \node[on chain, circle, scale=0.6, fill] (n0) {};
    \foreach \c [count=\x from 0] [count=\y from 1] in {A,BBBC,C,D} % Put Content in here
    {
      \node[on chain, rectangle split, rectangle split parts=2, rectangle split horizontal, draw] (\y) {\c};
      \node[on chain, circle, scale=0.6, fill, xshift=-22mm] (n\y){};
      \path[draw, thick, -latex', xshift=8.8mm] (n\x) -- (\y);
    }
    \node[on chain, rectangle, xscale=0.2, fill] (e) {};
    \path[draw, thick, xshift=8.8mm] (n\y) -- (e);
  \end{tikzpicture}
  \end{figure}

  \section{Fancy Code Block}
 
  \begin{lstlisting}
    Your code here
  \end{lstlisting}
\end{document}
