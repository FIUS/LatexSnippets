%Copyright (C) Tim Neumann
%Licensed under CC-BY-NC-SA
%Init
\usepackage{tikz}

\usetikzlibrary{shapes}
\usetikzlibrary{chains}
\usetikzlibrary{arrows}

%In Document
\begin{figure}[!h]
\centering
\begin{tikzpicture}[start chain, node distance=10mm]
  \node[on chain, circle, scale=0.6, fill] (n0) {};
  \foreach \c [count=\x from 0] [count=\y from 1] in {A,BBBC,C,D} % Put Content in here
  {
    \node[on chain, rectangle split, rectangle split parts=2, rectangle split horizontal, draw] (\y) {\c};
    \node[on chain, circle, scale=0.6, fill, xshift=-22mm] (n\y){};
    \path[draw, thick, -latex', xshift=8.8mm] (n\x) -- (\y);
  }
  \node[on chain, rectangle, xscale=0.2, fill] (e) {};
  \path[draw, thick, xshift=8.8mm] (n\y) -- (e);
\end{tikzpicture}
\end{figure}

